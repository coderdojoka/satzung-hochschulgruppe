% !TEX program = xelatex
\documentclass[a4paper, parskip=half, numbers=noenddot]{scrartcl}

\usepackage{a4wide}
\usepackage{geometry}
\geometry{left=25mm, right=25mm, top=23mm, bottom=33mm}

\usepackage[ngerman]{babel}
\usepackage[T1]{fontenc}

\usepackage{fontspec} % provides font selecting commands
\usepackage{xunicode} % provides unicode character macros
\usepackage{xltxtra}  % provides some fixes/extras

\usepackage{microtype}

\usepackage[juratotoc,ref=parlong]{scrjura}

\usepackage{hyperref}

% setzt einen kleinen Abstand \, zwischen Zahl und Buchstabe bei Paragraphen; ist so gewünscht
\renewcommand*{\thecontractSubParagraph}{%
{\theParagraph\,\alph{contractSubParagraph})}}

\renewcommand{\numberline}[1]{\makebox[2.5em][l]{#1}}


%
% Metadaten
%

\title{Satzung der Hochschulgruppe\\CoderDojo Karlsruhe}
\author{Version 0.3}
\date{Stand: 10. April 2015}
\hypersetup{
    pdftitle={Satzung der Hochschulgruppe CoderDojo Karlsruhe},
    pdfauthor={Hochschulgruppe CoderDojo Karlsruhe},
    pdfborder={0 0 0.5},
    linkbordercolor={0 0.61 0.50}
}


\begin{document}


%
% Titelseite
%

\maketitle
\vfill
\begin{contract}

%
% Inhaltsverzeichnis
%

\tableofcontents
\vspace{2em}

%
% Präambel
%

\section*{Präambel}

CoderDojo ist ein weltweites Netzwerk ehrenamtlich und gemeinschaftlich
betriebener Programmierklubs für Kinder und Jugendliche.
CoderDojo möchte den Teilnehmern eine lockere und ungezwungene Atmosphäre
bieten, um zu lernen Websites, Apps, Spiele und Programme zu entwickeln sowie im
Allgemeinen um zu lernen, moderne Informationstechnologie kreativ einzusetzen.
Gleichzeitig ist CoderDojo ein Platz für die Teilnehmer um Gleichgesinnte zu
treffen und sich gegenseitig auszutauschen.

Die Teilnehmer von CoderDojo werden von Mentoren ans Programmieren herangeführt,
und gleichzeitig angeleitet, anhand ihres Wissensstandes selbstständig weiter zu
lernen sowie dazu motiviert, sich gegenseitig auszutauschen, zu helfen und
Wissen weiter zu geben.
CoderDojo möchte dadurch den Teilnehmern das Programmieren als gemeinschaftliche
und soziale Aktivität präsentieren, die Spaß macht, und sie befähigen, die
erlernten Technologien zum Gestalten der Welt einzusetzen.

Dies ist die Satzung der Hochschulgruppe CoderDojo Karlsruhe, die CoderDojo
am Karlsruher Institut für Technologie (KIT) organisiert.
Die Hochschulgruppe verpflichtet sich, die Regelungen der
\href{http://www.asta-kit.de/studierendenschaft/ordnungen}{Hochschulgruppenordnung der Studierendenschaft am KIT}\footnote{Zu finden unter \url{http://www.asta-kit.de/sites/www.asta-kit.de/files/Hochschulgruppenordnung.pdf} bzw. verlinkt unter \url{http://www.asta-kit.de/studierendenschaft/ordnungen}} einzuhalten.
Aus Gründen der besseren Lesbarkeit wird ausschließlich die männliche Form
verwendet. Dabei ist jede andere Form impliziert. Die Geschlechtsdefinition
obliegt jeder Person selbst.
%\vspace{1em}
\newpage

%
% Die Hochschulgruppe
%

\Paragraph{title = Zweck der Hochschulgruppe}

Die Hochschulgruppe CoderDojo Karlsruhe organisiert CoderDojo am Karlsruher
Institut für Technologie.


%
% Mitgliedschaft
%

\Paragraph{title = Mitgliedschaft}

Die Mitgliedschaft beantragen kann jeder Mentor, der sich an einem von dieser
Hochschulgruppe organisierten CoderDojo beteiligt.

Beantragen der Mitgliedschaft
\begin{enumerate}
  \item Die Mitgliedschaft wird beim Vorstand beantragt und erfolgt durch
    Aufnahme in die Mitgliederliste.
  \item Der Vorstand beachtet bei der Bearbeitung der Mitgliedsanträge die in
    §2 Absatz (4) und Absatz (5) der Hochschulgruppenordnung definierten Quoten,
    und lehnt den Mitgliedsantrag gegebenenfalls ab:
    \begin{itemize}
      \item Die ordentlichen Mitglieder der Hochschulgruppe müssen zu mindestens 50\% am KIT immatrikuliert sein.
      \item Die ordentlichen Mitglieder der Hochschulgruppe müssen zu mindestens 75\% an einer Hochschule in Karlsruhe oder einer Partnerhochschule des KIT immatrikuliert sein oder sich dort in einem Ausbildungsverhältnis befinden.
    \end{itemize}
\end{enumerate}

Ende der Mitgliedschaft
\begin{enumerate}
  \item Die Mitgliedschaft endet durch Austritt, der gegenüber des Vorstands
    erklärt wird und durch Streichen von der Mitgliederliste erfolgt.
  \item Sofern der Eintritt eines Mitglieds erfolgte, während es zu einer der
    in §2 Absatz (4) und Absatz (5) der Hochschulgruppenordnung definierten
    Gruppe gehörte, endet die Mitgliedschaft, sobald das Mitglied diese
    Zugehörigkeit verliert.
    Die Mitgliedschaft kann jedoch sofort wieder nach §2 Absatz (2) beantragt
    werden.
\end{enumerate}
% TODO: Der Vorstand überprüft die Quoten rechtzeitig bevor die jährliche Rückmeldung
% der Hochschulgruppe ansteht.


Die Mitglieder arbeiten auf ehrenamtlicher Basis.

Die Mitglieder der Hochschulgruppe zahlen keine Beiträge.

%
% Organe
%

\Paragraph{title = Organe}

Die Organe der Hochschulgruppe sind

  \begin{enumerate}
  \item die Mitgliederversammlung und
  \item der Vorstand.
  \end{enumerate}


%
% Mitgliederversammlung
%

\Paragraph{title = Mitgliederversammung}%

Die Mitgliederversammlung ist das oberste Organ der Hochschulgruppe,
sie kann alle Angelegenheiten an sich ziehen und Beschlüsse des Vorstands
aufheben. Die Mitgliederversammlung hat besonders die folgenden Zuständigkeiten:
\begin{enumerate}
  \item die Bearbeitung und Beschlussfassung eingebrachter Anträge,
  \item Beschluss und Änderung der Satzung der Hochschulgruppe,
  \item Wahl und Entlastung des Vorstands,
  \item die Wahl zweier Mitglieder, die für die Kassenprüfung zuständig sind.
    Diese dürfen nicht dem Vorstand angehören und werden für die Dauer einer
    Abrechnungsperiode gewählt.
\end{enumerate}

Jedes Mitglied ist auf der Mitgliederversammlung stimm- und antragsberechtigt.

Die Mitgliederversammlung wird vom Vorstand mindestens einmal pro Semester sowie
auf Antrag von mindestens 3 Mitgliedern einberufen.
Bei der Einberufung muss eine Tagesordnung vorgeschlagen werden.
Wahlen und Abwahlen, sowie Anträge zur Änderung der Satzung müssen in dieser
Tagesordnung angekündigt werden.

Die Mitgliederversammlung muss mindestens 7 Tage im Voraus angekündigt
werden.

Die Mitgliederversammlung ist beschlussfähig, wenn ordnungsgemäß eingeladen
wurde und mindestens 5 Mitglieder anwesend sind.

Den Vorsitz in der Mitgliederversammlung führt der Vorstand.

Zur Durchführung der Wahlen zum Vorstand wählt die Mitgliederversammlung ein
Mitglied, das die Wahl leitet.
Dieses Mitglied darf nicht für den Vorstand kandidieren oder gewählt werden.

Vor Durchführung von Wahlen stellt das die Wahl leitende Mitglied die
Beschlussfähigkeit fest.

Die Mitgliederversammlung beschließt mit relativer Mehrheit.
Änderungen und Beschlüsse zur Satzung müssen mit Zweidrittelmehrheit der
anwesenden Mitglieder beschlossen werden.

Nach der Mitgliederversammlung ist den Mitgliedern durch den Vorstand ein
Protokoll zuzusenden, das die Anwesenheitsliste sowie alle behandelten
Beschlüsse samt Abstimmungsergebnis enthält.

% Paragraph Fachschaftsversammlung soll noch vollständig auf die Seite passen
%\enlargethispage*{\baselineskip}
%\pagebreak


%
% Vorstand
%

\Paragraph{title = Vorstand}

Der Vorstand ist das ausführende Organ der Hochschulgruppe, ihm ist die
Geschäftsführung der Hochschulgruppe im Sinne der Satzung übertragen.
Der Vorstand ist an die Beschlüsse der Mitgliederversammlung gebunden.

Der Vorstand besteht aus dem 1. und 2. Vorsitzenden.
Der 2. Vorsitzende vertritt den 1. Vorsitzenden bei dessen Abwesenheit.

Die Mitglieder des Vorstands werden von der Mitgliederversammlung für die Dauer
eines Jahres in getrennten Wahlgängen gewählt.

Die Amtszeit eines Vorstandsmitglieds endet
\begin{enumerate}
  \item nach Ablauf eines Jahres,
  \item durch eigenen Verzicht, oder
  \item durch Neuwahl.
\end{enumerate}
Bei Rücktritt eines Vorstandsmitglieds soll das verbleibende Vorstandsmitglied
zeitnah eine Mitgliederversammung einberufen um den vakanten Posten neu zu
besetzen.

Der Vorstand hat besonders die folgenden Aufgaben:
\begin{enumerate}
  \item die Einberufung und Leitung der Mitgliederversammlung,
  \item die Bearbeitung von Mitgliedsanträgen und die Führung der
    Mitgliederliste,
  \item die Veröffentlichung der Satzung,
  \item die Verantwortung für die Umsetzung von Beschlüssen tragen,
  \item als Ansprechpartner die Beschlüsse und Standpunkte der Hochschulgruppe
    kommunizieren,
  \item die Verwaltung der Finanzen der Hochschulgruppe, sowie
  \item die jährliche Rückmeldung der Hochschulgruppe beim Vorstand der
    Studierendenschaft am KIT.
\end{enumerate}

Der Vorstand ist der Mitgliederversammlung rechenschaftspflichtig.

Die Mitglieder des Vorstands arbeiten auf ehrenamtlicher Basis.


%
% Finanzen
%

\Paragraph{title = Finanzen}

Die Hochschulgruppe deckt ihre Aufwendungen durch Spenden und sonstige
Einnahmen.

Die Hochschulgruppe ist weder gewerblich noch eigenwirtschaftlich tätig.

Der Vorstand ist mit der Führung eines Kassenbuchs beauftragt.

Die für die Kassenprüfung zuständigen Mitglieder haben das Recht, jederzeit
Einsicht in die Buchhaltung zu nehmen.
Der Vorstand hat ihnen dabei Unterstützung zu gewähren.
Die Kassenprüfer haben die Pflicht, die Buchhaltung vor Ablauf der Wahlperiode
zu überprüfen.
Sie unterrichten die Mitgliederversammlung vom Ergebnis der Überprüfung.


%
% Schlussbestimmung
%

\Paragraph{title = Schlussbestimmung}

Diese Satzung tritt am \dots{} in Kraft.

\end{contract}
\end{document}
