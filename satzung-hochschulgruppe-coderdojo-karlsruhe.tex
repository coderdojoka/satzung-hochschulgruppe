% !TEX program = xelatex
\documentclass[a4paper, parskip=half, numbers=noenddot]{scrartcl}

\usepackage{a4wide}
\usepackage{geometry}
\geometry{left=25mm, right=25mm, top=23mm, bottom=33mm}

\usepackage[ngerman]{babel}
\usepackage[T1]{fontenc}

\usepackage{fontspec} % provides font selecting commands
\usepackage{xunicode} % provides unicode character macros
\usepackage{xltxtra}  % provides some fixes/extras

\usepackage{microtype}

\usepackage[juratotoc,ref=parlong]{scrjura}

\usepackage{hyperref}

% setzt einen kleinen Abstand \, zwischen Zahl und Buchstabe bei Paragraphen; ist so gewünscht
\renewcommand*{\thecontractSubParagraph}{%
{\theParagraph\,\alph{contractSubParagraph})}}

\renewcommand{\numberline}[1]{\makebox[2.5em][l]{#1}}


%
% Metadaten
%

\title{Satzung der Hochschulgruppe\\CoderDojo Karlsruhe}
\author{Version 0.1}
\date{Stand: 2. April 2015}
\hypersetup{
    pdftitle={Satzung der Hochschulgruppe CoderDojo Karlsruhe},
    pdfauthor={Hochschulgruppe CoderDojo Karlsruhe},
    pdfborder={0 0 0.5},
    linkbordercolor={0 0.61 0.50}
}


\begin{document}


%
% Titelseite
%

\maketitle
\thispagestyle{empty}

\pagestyle{empty}
\newpage
% Für Printversion eine leere Seite
\rule{0mm}{0mm}
\newpage


\begin{contract}

\setcounter{page}{1}
\pagestyle{plain}


%
% Inhaltsverzeichnis
%

\tableofcontents
\newpage


%
% Präambel
%

\section*{Präambel}

CoderDojo ist ein weltweites Netzwerk ehrenamtlich und gemeinschaftlich
betriebener Programmierklubs für Kinder und Jugendliche.
CoderDojo möchte den Teilnehmern eine lockere und ungezwungene Atmosphäre
bieten, um zu lernen Websites, Apps, Spiele und Programme zu entwickeln sowie im
Allgemeinen um zu lernen, moderne Informationstechnologie kreativ einzusetzen.
Gleichzeitig ist CoderDojo ein Platz für die Teilnehmer um Gleichgesinnte zu
treffen und sich gegenseitig auszutauschen.

Die Teilnehmer von CoderDojo werden von Mentoren ans Programmieren heran geführt,
und gleichzeitig angeleitet, anhand ihres Wissensstandes selbstständig weiter zu
lernen sowie dazu motiviert, sich gegenseitig auszutauschen, zu helfen und Wissen
weiter zu geben.
CoderDojo möchte dadurch den Teilnehmern das Programmieren als gemeinschaftliche,
soziale Aktivität präsentieren, die Spaß macht und sie befähigen, die erlernten
Technologien zum Gestalten der Welt einzusetzen.

Dies ist die Satzung der Hochschulgruppe CoderDojo Karlsruhe, die CoderDojo
am Karlsruher Institut für Technologie organisiert.
Aus Gründen der besseren Lesbarkeit wird ausschließlich die männliche Form verwendet.
Dabei ist jede andere Form impliziert.
Die Geschlechtsdefinition obliegt jeder Person selbst.


%
% Die Hochschulgruppe
%

\Paragraph{title = Zweck der Hochschulgruppe}

Die Hochschulgruppe CoderDojo Karlsruhe organisiert CoderDojo am Karlsruher
Institut für Technologie.

Die Hochschulgruppe ist weder gewerblich noch eigenwirtschaftlich tätig.


%
% Mitgliedschaft
%

\Paragraph{title = Mitgliedschaft}%

Mitglied werden kann jeder Mentor, der sich an einem von dieser Hochschulgruppe
organisierten CoderDojo beteiligt.

Die Aufnahme als Mitglied erfolgt durch Aufnahme in die Mitgliederliste durch
den Vorstand.

Die Mitgliedschaft endet durch Austritt, der gegenüber des Vorstands erklärt wird
und durch Streichen von der Mitgliederliste erfolgt.

Die Mitglieder arbeiten auf ehrenamtlicher Basis.

Die Mitglieder der Hochschulgruppe zahlen keine Beiträge.

%
% Organe
%

\Paragraph{title = Organe}

Die Organe der Hochschulgruppe sind

  \begin{enumerate}
  \item die Mitgliederversammlung und
  \item der Vorstand.
  \end{enumerate}


%
% Mitgliederversammlung
%

\Paragraph{title = Mitgliederversammung}%

Die Mitgliederversammlung ist das beschließende Organ der Hochschulgruppe und hat besonders die folgenden Zuständigkeiten:
\begin{enumerate}
\item Beschluss und Änderung der Satzung der Hochschulgruppe,
\item Wahl und Entlastung des Vorstands.
\end{enumerate}

Jedes Mitglied ist auf der Mitgliederversammlung stimm- und antragsberechtigt.

Die Mitgliederversammlung wird vom Vorstand mindestens einmal pro Semester sowie auf Antrag von mindestens drei Mitgliedern einberufen. Bei der Einberufung muss eine Tagesordnung vorgeschlagen werden.

Die Mitgliederversammlung muss mindestens 7 Tage im Voraus angekündigt werden.

Die Mitgliederversammlung ist beschlussfähig, wenn ordnungsgemäß eingeladen wurde und mindestens 5 Mitglieder anwesend sind.

Den Vorsitz in der Mitgliederversammlung führt der Vorstand.

Die Mitgliederversammlung beschließt mit relativer Mehrheit. Änderungen und Beschlüsse zur Satzung müssen mit Zweidrittelmehrheit der anwesenden Mitglieder beschlossen werden.

Nach der Mitgliederversammlung ist den Mitgliedern durch den Vorstand ein Protokoll zuzusenden.

% Paragraph Fachschaftsversammlung soll noch vollständig auf die Seite passen
%\enlargethispage*{\baselineskip}
%\pagebreak


%
% Vorstand
%

\Paragraph{title = Vorstand}%

Der Vorstand ist das ausführende Organ der Hochschulgruppe und vertritt sie nach außen.

Der Vorstand besteht aus dem 1. und 2. Vorsitzenden. Der 2. Vorsitzende vertritt den 1. Vorsitzenden bei dessen Abwesenheit.

Die Mitglieder des Vorstands werden von der Mitgliederversammlung für die Dauer eines Jahres gewählt.

Die Amtszeit eines Vorstandsmitglieds endet
\begin{enumerate}
  \item nach Ablauf eines Jahres,
  \item durch eigenen Verzicht, oder
  \item durch Neuwahl.
\end{enumerate}
Bei Rücktritt eines Vorstandsmitglieds soll das verbleibende Vorstandsmitglied zeitnah eine Mitgliederversammung einberufen um den vakanten Posten neu zu besetzen.

Aufgaben
\begin{enumerate}
\item Einberufung der Mitgliederversammlung,
\item Führung der Mitgliederliste,
\item Veröffentlichung der Satzung,
\item Verantwortung für die Umsetzung von Beschlüssen tragen,
\item als Ansprechpartner die Beschlüsse und Standpunkte der Hochschulgruppe kommunizieren, sowie die
\item Verwaltung der Finanzen der Hochschulgruppe.
\end{enumerate}

Der Vorstand ist der Mitgliederversammlung rech\-en\-schafts\-pflichtig.

Die Mitglieder des Vorstands arbeiten auf ehrenamtlicher Basis.


%
% Schlussbestimmung
%

\Paragraph{title = Schlussbestimmung}

Diese Satzung tritt am \dots{} in Kraft.

\end{contract}
\end{document}
